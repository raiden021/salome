\chapter{Fehlerdiskussion}
\section{Energiemessung}

In unserer ersten Messung haben wir die kin.Energie der Elektronen mithilfe 
des linearen Zusammenhanges zwischen Ablenkung des Strahls im homogenen Magnetfeld
und der Magnetfeldst�rke. Mithilfe der linearen Regression und Fehlerfortpflanzung.
Jede Messung hatte eine Messungenauigkeit durch den Apparat den wir zur Aufnahme der 
Strahlposition genommen haben von 0.25 mm. Da diese Ungenauigkeit bei jedem Messpunkt 
gleich war haben wir keine Gewichtung der einzelnen Messpunkte vorgenommen.

Das Messergebnis war : $E_\text{kin}$=(7.2 $\pm$ 0.2) keV

Ein systematischer Fehler bei der Messung k�nnte die statistische Verteilung der 
kin.Energie der Teilchen durch die thermische Erzeugung in der Kathode sein.
Teilchen mit verschiedenen Energien w�rden auch anders im Magnetfeld abgelenkt,
jedoch w�rde das in erster Linie zu einer Verbreiterung des Strahles in 
y Richtung f�hren, was wir nicht beobachten konnten.
Nichts desto trotz k�nnte man zur Verbesserung der Messung eine genauer
Elektronenstrahlquelle verwenden.


\section{Erste und zweite Emittanzmessung}

Die Emittanz hatten wir auf zwei verschiedene Arten gemessen.
Einmal durch Messung der Strahlgr��e bei Variation des Quadropolstromes an einem Schirm 
und einmal die gleiche Messung aber an 4 verschiedenen Schirmen.
Wir haben eine Parabel durch einen $\chi ^2$ Fit angen�hert, und
mit den 3 Parametern der Parabel die Emittanz mit Fehlerfortpflanzung
errechnet.

Dabei f�llt nat�rlich auf, dass eine Messung mit 3 Freiheitsgraden und nur 4 Messpunkten wie es
bei Messung 2 der Fall ist eine statistische Fehleranalyse mit einem $\chi ^2$ Fit wenig Aussagekraft hat.

Messung 1: \begin{equation}\eps_{x,n} = (0,136 \pm 0,005)\text{ mm}\cdot\text{mrad.} \nonumber \end{equation}
            \begin{equation}\eps_{y,n} = (0,222 \pm 0,011)\text{ mm}\cdot\text{mrad.} \nonumber\end{equation}

Messung 2: \begin{equation}\eps_{x,n} = (0,119 \pm 0,016)\text{ mm}\cdot\text{mrad} \nonumber \end{equation}
            \begin{equation}\eps_{y,n} = (0,35 \pm 0,02)\text{ mm}\cdot\text{mrad.} \nonumber\end{equation}           

In der Matrizenrechnungen gehen wir davon aus das die Dipole �u�ere St�rfelder kompensiert haben, sodass wir weder das eine noch das andere ber�cksichtigen m�ssen.
In der Realit�t ist dies nat�rlich nicht der Fall, sodass besonders Abweichungen durch das Erdmagnetfeld in unseren Matrizen nicht ber�cksichtigt werden konnten. 
In unsere zweiten Messung zur Emittanz haben wir an verschiedenen Schirmen gemessen, sodass sich die Fehler aufsummiert haben und in der Fehlerfortpflanzung zu gro�en Fehler f�hrten.

Geht man davon aus, dass das Erdmagnetfeld eine gro�e Komponente vertikal zur Sollbahn hat, erwartet man ein gro�es St�rfeld in horizontaler x-Richtung, was unsere gro�e Standardabweichung bei der Messung erkl�ren k�nnte.
Eine Idee zur Eind�mmung dieses Problemes w�re die Messung mit verschiedenen Quadropol Einstellung durchzuf�hren und gucken bei welchem die Standardabweichung der  
Emittanz sinkt. Die Matrix ist Teil der Fehlerrechnung und kann, wenn sie passend gew�hlt wird, zu einer Fehlerfortpflanzung mit weniger gro�en Fehlerwerten f�hren.

\section{Beam-based Alignment}

In dem ``Beam-based Alignment'' haben wir an einem Schirm Strahlposition bei variierenden Dipolablenkung gemessen. 

F�r 2 verschiedene Quadropolstr�me haben wir die Messwerte aufgetragen und geguckt wo sich die Geraden schneiden.

x-Dipol I=-3.055A

y-Dipol I= 2.211A

Um die Werte zu testen haben wir die Dipole auf diese Werte eingestellt und den Quadropol von -2A bis 2A laufen lassen 
und geguckt ob sich die Strahlmitte verschiebt.

Die Verschiebung fand im Rahmen der Messungenauigkeit statt, sodass wir davon ausgehen k�nnen, dass wir sehr gute Werte erreicht haben.

