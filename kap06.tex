\chapter{}
\section{}
Vor den eigentlichen Experimenten muss man die Optik einstellen, sodass der Strahl am Ende des Beschleunigers ohne gro�e Teilchenverluste mittig ankommt.
Wir haben bei gegebenen Quadropolstromwerten, die den Strahl fokussieren, die Dipolstr�me so angepasst das der Strahl jeweils mittig durch die 5 Schirme gelangt.

Vor jedem Schirm ist ein Dipolpaar justiert, sodass wir den Strahl in x und y Richtung bewegen k�nnen. 

Im ersten Schirm haben wir den Strahl direkt nach dem verlassen der Kathode gemessen. 

Wie erwartet war der Strahl schon durch die Kathode gut fokussiert. Um zu verfizieren, dass wir wirklich den Strahl auf dem Schirm sehen und keine Reflexionen, haben wir mit
dem Magnetfeld eines Handys den Strahl abgelenkt und tats�chlich konnte man gut sehen wie stark so ein kleines St�rfeld den Strahl umlenkt.

Als n�chstes haben wir an jedem Schirm den horizontalen und vertikalen Dipolmagneten in 1A Schritten von -3A bis 3A durchfahren bis wir den Strahl auf dem n�chsten Schirm
sehen konnten. Der Strahl war jedesmal gut fokussiert, sodass wir den Quadropolstrom nicht anpassen mussten.

Die Aufnahmen der 5 Schirme mit den dazugeh�rigen Dipolstromwerten sind unten aufgef�hrt.

\begin{figure}[ht]
  \subfigure[]{
    \includegraphics[width=0.3\textwidth]{}
    }
  \hfill
  \subfigure[]{
    \includegraphics[width=0.3\textwidth]{}
    }
  \hfill
  \subfigure[]{
    \includegraphics[width=0.3\textwidth]{}
    }\\
  \subfigure[]{
    \includegraphics[width=0.3\textwidth]{}
    }
  \hfill
  \subfigure[]{
    \includegraphics[width=0.3\textwidth]{}
   }
  \caption{Aufnahmen des Strahles auf den  Schirmen 1-5}
  \label{abb:portraits}
\end{figure}