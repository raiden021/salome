\chapter{Energiemessung}

Zun�chst wollen wir eine sogenannte Energiemessung des
Elektronenstrahls mithilfe
eines Dipolmagneten durchf�hren. Die
Elektronen, aus dem der Strahl besteht, werden
am Anfang des Beschleunigers thermisch durch die Kathode erzeugt
und erhalten so ihre kinetische Energie und ihren Impulsbetrag,
die sie von da an behalten, da sie im Beschleuniger nur mit den
Feldern der Dipol- und Quadrupolmagnete �ber die Lorentzkraft
wechselwirken.

Dies geschieht, indem der Zusammenhang zwischen
Ablenkung durch einen der Dipole und St�rke des Dipols gemessen
wird, da erstere theoretisch linear von der letzteren abh�ngt.
Genauer gilt:

Sei $L_{eff}$ die L�nge des Einflussbereichs des Dipols, sei dahinter
eine Driftstrecke mit L�nge $L_{drift}$ und am Ende ein Schirm. Sei
$x$ der Versatz des Strahls auf dem Schirm in die Richtung,
in die er vom Dipol abgelenkt wird. Es gilt dann
$x = x_0 + \frac{dx}{dI} I$, wobei $x_0$ der Versatz auf dem Schirm
ohne Wirkung des Dipols ist. (Man bedenke, dass $\frac{dx}{dI}$ negativ
sein kann, falls $x_0$ so ist, dass die Ablenkung durch den Dipol
den Strahl n�her in die Mitte des Strahlrohrs versetzt.) $I$ ist der
Strom des Dipols, der proportional zur Dipolst�rke
mit Faktor $\kappa$ ist. Sei $v$ die Geschwindigkeit der auf den
Schirm treffenden Elektronen, $\beta := \frac{v}{c}$, $c$ die Lichtgeschwindigkeit und $\gamma := \frac{1}{\sqrt{1-\beta^2}}$.
Dann gilt:

\begin{equation} \label{eq:dipolabl}
	\beta \gamma = \frac{e L_{eff} L_{drift} }
	                    {\kappa m_e c |\frac{dx}{dI}|}
\end{equation}

Wir messen eine Reihe von Wertepaaren f�r $I$ und $x$, ermitteln mithilfe
linearer Regression einen Wert f�r $\frac{dx}{dI}$ und erhalten
daraus mit \eqref{eq:dipolabl} einen Wert f�r $\beta \gamma$.

\begin{table}[ht]
\centering
\begin{tabular}{| >{$}c<{$} >{$}c<{$} |}
\hline
I [A]	&		x [mm]	 \\ [0.2ex]
\hline\hline
-3,50	&		4,07	 \\
-3,45	&		3,41	 \\
-3,40	&		2,71	 \\
-3,35	&		2,05	 \\
-3,30	&		1,23	 \\
-3,25	&		0,37	 \\
-3,20	&		-0,10	 \\
-3,15	&		-0,58	 \\
-3,10	&		-1,44	 \\
-3,05	&		-2,19	 \\
-3,00	&		-2,85	 \\
-2,95	&		-3,55	 \\
-2,90	&		-4,16	 \\
-2,85	&		-5,05	 \\
-2,80	&		-6,21	 \\
-2,75	&		-6,82	 \\
-2,70	&		-7,85	 \\
-2,65	&		-8,56	 \\
-2,60	&		-9,21	 \\
-2,55	&		-9,78	 \\
-2,50	&		-10,17	 \\
\hline
\end{tabular}
\caption{Messung des horizontalen Versatzes $x$ auf dem Schirm }
\end{table}