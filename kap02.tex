\chapter{Energiemessung}
\section{}

Zun�chst wollen wir die sogenannte Energiemessung des
Elektronenstrahls durchf�hren. Die
Elektronen, aus dem der Strahl besteht, werden
am Anfang des Beschleunigers thermisch durch die Kathode erzeugt
und erhalten so ihre kinetische Energie und ihren Impulsbetrag,
die sie von da an behalten, da sie im Beschleuniger nur mit den
Feldern der Dipol- und Quadrupolmagnete �ber die Lorentzkraft
wechselwirken.

Dies geschieht, indem der Zusammenhang zwischen
Ablenkung durch einen der Dipole und St�rke des Dipols gemessen
wird, da erstere theoretisch linear von der letzteren abh�ngt.
Genauer gilt:

Sei $L_{eff}$ die L�nge des Einflussbereichs des Dipols, sei dahinter
eine Driftstrecke mit L�nge $L_{drift}$ und am Ende ein Schirm. Sei
$x$ der Versatz des Strahls auf dem Schirm in die Richtung,
in die er vom Dipol abgelenkt wird, und $\frac{dx}{dI}$ die
Proportionalit�tskonstante zwischen $x$ und $I$, wobei $I$ der
Strom des Dipols ist, der n�mlich selber proportional zur Dipolst�rke
ist. Sei $v$ die Geschwindigkeit der auf den Schirm treffenden
Elektronen, $\beta := \frac{v}{c}$ mit der Lichtgeschwindigkeit $c$
und $\gamma := \frac{1}{\sqrt{1-\beta^2}}$. Dann gilt:


\begin{equation}
	\beta \gamma = \frac{e \kappa L_{drift}}{m_e c \frac{dx}{dI}}
\end{equation}