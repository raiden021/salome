\chapter{Emittanzmessung}

An jedem Punkt $z$ entlang des Orbits hat der Elektronenstrahl
unter Anderem die Gr��en $\langle x^2\rangle(z)$, $\langle xx'\rangle(z)$,
$\langle x'^2\rangle(z)$, bzw.
$\langle y^2\rangle(z)$, $\langle yy'\rangle(z)$, $\langle y'^2\rangle(z)$. Wenn
$R = \bigl(\begin{smallmatrix} R_{11}&R_{12}\\ R_{21}&R_{22}
\end{smallmatrix} \bigr)$ die Transformationsmatrix der Gr��en $x$
und $x'$ eines einzelnen Elektrons von $z_0$ zu $z_1$ ist,
dann gilt
\begin{equation}
	\langle x^2\rangle(z_1) =
	\begin{pmatrix} R_{11}^2 & 2R_{12}R_{11} & R_{12}^2 \end{pmatrix}
	\begin{pmatrix} \langle x^2\rangle(z_0)\\\langle xx'\rangle(z_0)
					\\\langle x'^2\rangle(z_0) \end{pmatrix} \text{,}
\end{equation}
und es gilt Analoges f�r $y$.

F�r die entlang des Orbits erhaltene Emittanz $\eps$ gilt
\begin{equation}
		\eps = \sqrt{\langle x^2\rangle(z)\langle x'^2\rangle(z)
		- \langle xx'\rangle(z)^2} \quad \forall z
\end{equation}

Wir werden hierzu die Gr��en $a_1 := \langle x^2\rangle$, $a_2 := \langle
xx'\rangle$ und
$a_3 := \langle x'^2\rangle$ (und alles Folgende analog f�r $y$) direkt vor dem
Quadrupolmagneten
V46MATCH bestimmen und daraus dann die transversalen Emittanzen $\eps_x$ und
$\eps_y$ und ihre normierten Varianten $\eps_{x,n}$ und $\eps_{y,n}$ erhalten.

F�r ersteres werden wir den Elektronenstrahl $n$-mal auf verschiedene Weise
transformieren (�ber verschiedene Strecken oder Optiken), wobei $b_i$
die Gr��e $\langle x^2\rangle$ nach der $i$-ten Transformation und
$R^{(i)}$ die entsprechende Transformationsmatrix ist.
Das f�r gro�es $n$ �berbestimmte lineare Gleichungssystem
\begin{equation}
 \begin{pmatrix} b_1 \\ \vdots \\ b_n \end{pmatrix} = \underbrace{
 \begin{pmatrix} {R^{(1)}_{11}}^2 & 2R^{(1)}_{12}R^{(1)}_{11} & {R^{(1)}_{12}}^2 \\
				 & \vdots & \\
				 {R^{(n)}_{11}}^2 & 2R^{(n)}_{12}R^{(n)}_{11} & {R^{(n)}_{12}}^2 \\
 \end{pmatrix} }_{=:B}
 \begin{pmatrix} a_1 \\ a_2 \\ a_3 \end{pmatrix}
\end{equation}
an die zu ermitteltenden Unbekannten $a_1$, $a_2$, $a_3$ kann und wird dann
Grundlage f�r einen $\chi^2$-Fit sein.

\section{Messung durch Variation des Quadrupolstroms}

Zuerst lassen wir den Elektronenstrahl von dem Punkt direkt vor
V46MATCH bis zum Schirm 2 auf verschiedene Weise transformieren, indem
wir an V46MATCH jeweils 25, bzw. 19 verschiedene Stromst�rken einstellen und dann
$\langle x^2 \rangle$, bzw. $\langle y^2 \rangle$ an Schirm 3 messen. Die
Messergebnisse sind in den Tabellen \ref{tab:xemitquad} und \ref{tab:yemitquad}
dargestellt.

\begin{table}[ht]
\centering
\begin{tabular}{| >{$}c<{$} >{$}c<{$} |}
\hline
I\text{ [A]}	&	\langle x^2 \rangle \text{ [mm]}	 \\ [0.2ex]
\hline\hline
0,70	&	0,66	\\
0,75	&	0,616	\\
0,80	&	0,58	\\
0,85	&	0,55	\\
0,90	&	0,5	\\
0,95	&	0,48	\\
1,00	&	0,44	\\
1,05	&	0,4	\\
1,10	&	0,4	\\
1,15	&	0,39107	\\
1,20	&	0,382277	\\
1,25	&	0,381375	\\
1,30	&	0,380214	\\
1,35	&	0,396236	\\
1,40	&	0,403774	\\
1,45	&	0,429807	\\
1,50	&	0,450623	\\
1,55	&	0,466817	\\
1,60	&	0,500828	\\
1,65	&	0,525564	\\
1,70	&	0,548818	\\
1,75	&	0,572263	\\
1,80	&	0,627355	\\
1,85	&	0,659232	\\
1,90	&	0,673288	\\

\hline
\end{tabular}
\caption{Messung der horizontalen Strahlgr��e $\langle x^2 \rangle$ an Schirm 2
			in Abh�ngigkeit des Quadrupolstroms $I$ bei V46MATCH.}
	\label{tab:xemitquad}
\end{table}

\begin{table}[ht]
\centering
\begin{tabular}{| >{$}c<{$} >{$}c<{$} |}
\hline
I\text{ [A]}	&	\langle y^2 \rangle \text{ [mm]}	 \\ [0.2ex]
\hline\hline


\hline
\end{tabular}
\caption{Messung der vertikalen Strahlgr��e $\langle y^2 \rangle$ an Schirm 2
			in Abh�ngigkeit des Quadrupolstroms $I$ bei V46MATCH.}
	\label{tab:yemitquad}
\end{table}




\section{Messung durch Variation der Schirme}