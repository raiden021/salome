\chapter{Emittanzmessung}

An jedem Punkt $z$ entlang des Orbits hat der Elektronenstrahl
unter Anderem die Gr��en $\langle x^2\rangle(z)$, $\langle xx'\rangle(z)$,
$\langle x'^2\rangle(z)$, bzw.
$\langle y^2\rangle(z)$, $\langle yy'\rangle(z)$, $\langle y'^2\rangle(z)$. Wenn
$R = \bigl(\begin{smallmatrix} R_{11}&R_{12}\\ R_{21}&R_{22}
\end{smallmatrix} \bigr)$ die Transformationsmatrix der Gr��en $x$
und $x'$ eines einzelnen Elektrons von $z_0$ zu $z_1$ ist,
dann gilt
\begin{equation}
	\langle x^2\rangle(z_1) =
	\begin{pmatrix} R_{11}^2 & 2R_{12}R_{11} & R_{12}^2 \end{pmatrix}
	\begin{pmatrix} \langle x^2\rangle(z_0)\\\langle xx'\rangle(z_0)
					\\\langle x'^2\rangle(z_0) \end{pmatrix} \text{,}
\end{equation}
und es gilt Analoges f�r $y$.

F�r die entlang des Orbits erhaltene Emittanz $\eps$ gilt
\begin{equation}
		\eps = \sqrt{\langle x^2\rangle(z)\langle x'^2\rangle(z)
		- \langle xx'\rangle(z)^2} \quad \forall z
\end{equation}

Wir werden hierzu die Gr��en $a_1 := \langle x^2\rangle$, $a_2 := \langle
xx'\rangle$ und
$a_3 := \langle x'^2\rangle$ (genauso f�r $y$) direkt vor dem Quadrupolmagneten
V46MATCH bestimmen und daraus dann die transversalen Emittanzen $\eps_x$ und
$\eps_y$ und ihre normierten Varianten $\eps_{x,n}$ und $\eps_{y,n}$ erhalten.

F�r ersteres werden wir den Elektronenstrahl $n$-mal auf verschiedene Weise
transformieren (�ber verschiedene Strecken oder Optiken), wobei $b_i$
die Gr��e $\langle x^2\rangle$ nach der $i$-ten Transformation und
$R^{(i)}$ die entsprechende Transformationsmatrix ist.
Das f�r gro�es $n$ �berbestimmte lineare Gleichungssystem
\begin{equation}
 \begin{pmatrix} b_1 \\ \vdots \\ b_n \end{pmatrix} = 
 \begin{pmatrix} {R^{(1)}_{11}}^2 & 2R^{(1)}_{12}R^{(1)}_{11} & {R^{(1)}_{12}}^2 \\
				 & \vdots & \\
				 {R^{(n)}_{11}}^2 & 2R^{(n)}_{12}R^{(n)}_{11} & {R^{(n)}_{12}}^2 \\
	\end{pmatrix}
	\begin{pmatrix} a_1 \\ a_2 \\ a_3 \end{pmatrix}
\end{equation}
an die zu ermitteltenden Unbekannten $a_1$, $a_2$, $a_3$ ist dann Grundlage f�r
einen $\chi^2$-Fit.

\section{Messung durch Variation des Quadrupolstroms}

\section{Messung durch Variation der Schirme}