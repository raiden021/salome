\chapter{Emittanzmessung}

An jedem Punkt $s$ entlang des Orbits hat der Elektronenstrahl
unter Anderem die Gr��en $\langle x^2\rangle(s)$,
$\langle xx'\rangle(s)$ und $\langle x'^2\rangle(s)$, bzw.
$\langle y^2\rangle(s)$, $\langle yy'\rangle(s)$ und
$\langle y'^2\rangle(s)$. Wenn
$R = \bigl(\begin{smallmatrix} R_{11}&R_{12}\\ R_{21}&R_{22}
\end{smallmatrix} \bigr)$ die Transformationsmatrix der Gr��en $x$
und $x'$ eines einzelnen Elektrons von $s_0$ zu $s_1$ ist,
dann gilt
\begin{equation}
	\langle x^2\rangle(s_1) =
	\begin{pmatrix} R_{11}^2 & 2R_{12}R_{11} & R_{12}^2 \end{pmatrix}
	\begin{pmatrix} \langle x^2\rangle(s_0)\\\langle xx'\rangle(s_0)
					\\\langle x'^2\rangle(s_0) \end{pmatrix} \text{,}
\end{equation}
und es gilt Analoges f�r $y$.

F�r die entlang des Orbits erhaltene normierte Emittanz $\eps_{x,n}$ gilt
\begin{equation} \label{eq:normemit}
		\eps_{x,n}
		= \beta\gamma\sqrt{\langle x^2\rangle(s)\langle x'^2\rangle(s)
		  - \langle xx'\rangle(s)^2} \quad \forall s
\end{equation}

Wir werden hierzu die Gr��en $a_1 := \langle x^2\rangle$, $a_2 := \langle
xx'\rangle$ und
$a_3 := \langle x'^2\rangle$ (und alles Folgende analog f�r $y$) einmal direkt
vor dem
Quadrupolmagneten
V46MATCH und einmal an Schirm 2 bestimmen und daraus dann jeweils die
transversalen Emittanzen $\eps_x$ und
$\eps_y$ und ihre normierten Varianten $\eps_{x,n}$ und $\eps_{y,n}$ erhalten.

F�r ersteres werden wir den Elektronenstrahl $n$-mal auf verschiedene Weise
transformieren (�ber verschiedene Strecken oder Optiken), wobei $b_i$
die Gr��e $\langle x^2\rangle$ nach der $i$-ten Transformation und
$R^{(i)}$ die entsprechende Transformationsmatrix ist.
Das f�r gro�es $n$ �berbestimmte lineare Gleichungssystem
\begin{equation} \label{eq:transform}
 \begin{pmatrix} b_1 \\ \vdots \\ b_n \end{pmatrix} = \underbrace{
 \begin{pmatrix} {R^{(1)}_{11}}^2 & 2R^{(1)}_{12}R^{(1)}_{11} & {R^{(1)}_{12}}^2 \\
				 & \vdots & \\
				 {R^{(n)}_{11}}^2 & 2R^{(n)}_{12}R^{(n)}_{11} & {R^{(n)}_{12}}^2 \\
 \end{pmatrix} }_{=:B}
 \begin{pmatrix} a_1 \\ a_2 \\ a_3 \end{pmatrix}
\end{equation}
an die zu ermitteltenden Unbekannten $a_1$, $a_2$, $a_3$ kann und wird dann
Grundlage f�r einen $\chi^2$-Fit sein.

\section{Messung durch Variation des Quadrupolstroms}

Zuerst lassen wir den Elektronenstrahl von dem Punkt direkt vor
V46MATCH bis zum Schirm 2 auf verschiedene Weise transformieren, indem
wir an V46MATCH jeweils 25, bzw. 19 verschiedene Stromst�rken einstellen und dann
$\langle x^2 \rangle$, bzw. $\langle y^2 \rangle$ an Schirm 3 messen. Dazu
nehmen wir wieder zun�chst ein Bild des Hintergrunds (ohne Strahl), welches
wir danach von der Kameraaufnahme abziehen. Die Strahlgr��e $\sqrt{\langle x^2 \rangle}$
wird dann, nachdem manuell mit der Maus ein enger Bereich um den Strahlfleck
auf dem Schirm abgegrenzt wird, von dem Bildanalyse-Programm als Varianz der
Intensit�tsverteilung auf dem Kamerasensor berechnet. Die
Messergebnisse sind in den Tabellen \ref{tab:xemitquad} und \ref{tab:yemitquad}
dargestellt.

\begin{table}[ht]
\centering
\begin{tabular}{| >{$}c<{$} >{$}c<{$} |}
\hline
I\text{ [A]}	&	\sqrt{\langle x^2 \rangle} \text{ [mm]}	 \\ [0.2ex]
\hline\hline
0,70	&	0,66		\\
0,75	&	0,616		\\
0,80	&	0,58		\\
0,85	&	0,55		\\
0,90	&	0,5			\\
0,95	&	0,48		\\
1,00	&	0,44		\\
1,05	&	0,4			\\
1,10	&	0,4			\\
1,15	&	0,39107		\\
1,20	&	0,382277	\\
1,25	&	0,381375	\\
1,30	&	0,380214	\\
1,35	&	0,396236	\\
1,40	&	0,403774	\\
1,45	&	0,429807	\\
1,50	&	0,450623	\\
1,55	&	0,466817	\\
1,60	&	0,500828	\\
1,65	&	0,525564	\\
1,70	&	0,548818	\\
1,75	&	0,572263	\\
1,80	&	0,627355	\\
1,85	&	0,659232	\\
1,90	&	0,673288	\\

\hline
\end{tabular}
\caption{Messung der horizontalen Strahlgr��e $\sqrt{\langle x^2 \rangle}$ an Schirm 2
			in Abh�ngigkeit des Quadrupolstroms $I$ bei V46MATCH.}
	\label{tab:xemitquad}
\end{table}

\begin{table}[ht]
\centering
\begin{tabular}{| >{$}c<{$} >{$}c<{$} |}
\hline
I\text{ [A]}	&	\sqrt{\langle y^2 \rangle} \text{ [mm]}	 \\ [0.2ex]
\hline\hline
-1,90	&	0,710957	\\
-1,85	&	0,658355	\\
-1,80	&	0,609792	\\
-1,75	&	0,573970	\\
-1,70	&	0,533938	\\
-1,65	&	0,509606	\\
-1,60	&	0,490458	\\
-1,55	&	0,470464	\\
-1,50	&	0,462698	\\
-1,45	&	0,464849	\\
-1,40	&	0,469504	\\
-1,35	&	0,475496	\\
-1,30	&	0,497975	\\
-1,25	&	0,530275	\\
-1,20	&	0,561653	\\
-1,15	&	0,593314	\\
-1,10	&	0,632141	\\
-1,05	&	0,705271	\\
-1,00	&	0,733322	\\
\hline
\end{tabular}
\caption{Messung der vertikalen Strahlgr��e $\sqrt{\langle y^2 \rangle}$ an Schirm 2
			in Abh�ngigkeit des Quadrupolstroms $I$ bei V46MATCH.}
	\label{tab:yemitquad}
\end{table}

Mit einem Matlab-Skript wird dann jeweils aus den verschiedenen
Stromeinstellungen die Matrix $B$ aus Gleichung \eqref{eq:transform} bestimmt und
daraus der $\chi^2$-Fit f�r $a_1$, $a_2$, $a_3$ bestimmt. Mittels
Fehlerfortpflanzung erhalten wir dann nach \eqref{eq:normemit} einen Wert
und Fehler f�r $\eps_{x,n}$ und $\eps_{y,n}$.

In horizontale Richtung $x$ haben wir
\begin{equation}
	B =
	\begin{pmatrix}
	0,0267  &  0,1297  &  0,1572	\\
    0,0111  &  0,0832  &  0,1554	\\		
    0,0023  &  0,0375  &  0,1537	\\	
    0,0001  & -0,0075  &  0,1519	\\	
    0,0045  & -0,0519  &  0,1501	\\		
    0,0154  & -0,0955  &  0,1484	\\	
    0,0327  & -0,1385  &  0,1467	\\
    0,0563  & -0,1807  &  0,1449	\\
    0,0862  & -0,2223  &  0,1432	\\
    0,1223  & -0,2632  &  0,1416	\\
    0,1645  & -0,3034  &  0,1399	\\	
    0,2128  & -0,3430  &  0,1382	\\
    0,2670  & -0,3819  &  0,1366	\\
    0,3270  & -0,4201  &  0,1350	\\
    0,3929  & -0,4577  &  0,1333	\\
    0,4645  & -0,4947  &  0,1317	\\
    0,5417  & -0,5310  &  0,1301	\\
    0,6245  & -0,5667  &  0,1286	\\
    0,7128  & -0,6017  &  0,1270	\\
    0,8066  & -0,6361  &  0,1254	\\
    0,9057  & -0,6699  &  0,1239	\\
    1,0101  & -0,7031  &  0,1224	\\
    1,1197  & -0,7357  &  0,1208	\\
    1,2344  & -0,7676  &  0,1193	\\
    1,3542  & -0,7990  &  0,1178			
	\end{pmatrix} \nonumber
\end{equation}

wobei die zweite Spalte in m und die dritte in m$^2$ angegeben sind.
Der $\chi^2$-Fit ergibt daraus:
\begin{equation}
	\begin{pmatrix}
	a_1	\\
    a_2	\\
	a_3				
	\end{pmatrix}
	= 10^{-5} \cdot
	\begin{pmatrix}
	0,063 \text{ m}^2 \\
	0,0762 \text{ m} \\
	0,197			
	\end{pmatrix} \nonumber
\end{equation}

Schliesslich erhalten wir damit
\begin{equation}
	\eps_{x,n} = (0,136 \pm 0,005)\text{ mm}\cdot\text{mrad.} \nonumber
\end{equation}

F�r die vertikale Richtung $y$ haben wir analog
\begin{equation}
	B =
	\begin{pmatrix}
   	   1,3542   &	-0,7990     &	  0,1178			\\
       1,2344   &	-0,7676     &	  0,1193			\\		
       1,1197   &	-0,7357      &	 0,1208				\\
       1,0101   &	-0,7031      &	 0,1224				\\
       0,9057   &	-0,6699      &	 0,1239				\\
       0,8066     &	 -0,6361     &	  0,1254			\\
       0,7128     &	 -0,6017     &	  0,1270			\\
       0,6245     &	 -0,5667     &	  0,1286			\\
       0,5417     &	 -0,5310     &	  0,1301			\\
       0,4645     &	 -0,4947     &	  0,1317			\\
       0,3929     &	 -0,4577     &	  0,1333			\\
       0,3270     &	 -0,4201     &	  0,1350			\\
       0,2670     &	 -0,3819     &	  0,1366			\\
       0,2128     &	 -0,3430     &	  0,1382			\\
       0,1645     &	 -0,3034     &	  0,1399			\\
       0,1223    &	  -0,2632    &	   0,1416			\\
       0,0862    &	  -0,2223     &	  0,1432			\\
       0,0563    &	  -0,1807     &	  0,1449			\\
       0,0327    &	  -0,1385     &	  0,1467
	\end{pmatrix} \nonumber
\end{equation}

mit der zweiten Spalte in m und der dritten in m$^2$,

\begin{equation}
	\begin{pmatrix}
	a_1	\\
    a_2	\\
	a_3				
	\end{pmatrix}
	= 10^{-5} \cdot
	\begin{pmatrix}
	0,111  \text{ m}^2 \\
	0,2063 \text{ m} \\
	0,5421			
	\end{pmatrix} \nonumber
\end{equation}

und

\begin{equation}
	\eps_{y,n} = (0,222 \pm 0,011)\text{ mm}\cdot\text{mrad.} \nonumber
\end{equation}

\section{Messung durch Variation der Schirme}

Als n�chstes wird der Elektronenstrahl von Schirm 2
auf verschiedene Weise transformiert, indem alle Quadrupol- und Dipolstr�me
(die \emph{Optik}) konstant gehalten werden und daf�r die Strecken, die der
Strahl zur�cklegt, variiert werden. Wir messen hierzu die Strahlgr��en
$\langle x^2 \rangle$ und $\langle y^2 \rangle$ an den Schirmen 2, 3, 4 und 5.
Die Messergebnisse sind in Tabelle \ref{tab:emitschirm} dargestellt.

\begin{table}[ht]
\centering
\begin{tabular}{| >{$}c<{$} | >{$}c<{$} >{$}c<{$} >{$}c<{$} >{$}c<{$} |}
\hline
 & \text{Schirm 2} & \text{Schirm 3} & \text{Schirm 4} & \text{Schirm 5}\\[0.2ex]
\hline
\sqrt{\langle x^2\rangle} \text{ [mm]} & 0,9018 & 0,8594 & 1,0907 & 1,2662 \\
\sqrt{\langle y^2 \rangle} \text{ [mm]} & 1,0049 & 2,4088 & 2,9551 & 1,3854 \\
\hline
\end{tabular}
\caption{Messung der Strahlgr��en $\sqrt{\langle x^2 \rangle}$ und
		$\sqrt{\langle y^2 \rangle}$ an den Schirmen 2, 3, 4 und 5.}
		\label{tab:emitschirm}
\end{table}

Es gibt dann f�r jeden der vier Schirme eine Transformationsmatrix, die
den Elektronenstrahl zu diesem Schirm transformiert. (F�r Schirm 2 ist dies
die Identit�t.) Aus diesen vier Transformationsmatrizen wird durch Matlab
wieder jeweils die Matrix $B$ im Sinne von \eqref{eq:transform} berechnet:
\begin{equation}
	B_x =
	\begin{pmatrix}
	1  &       0   &      0	\\
	0,2188  &  0,4660   & 0,2481	\\	
	0,0673  & -0,6866   & 1,7499	\\	
	0,3917  & -2,0083   & 2,5742
	\end{pmatrix}  \nonumber
\end{equation}

f�r die horizontale Richtung $x$ und

\begin{equation}
	B_y =
	\begin{pmatrix}
	1  &       0  &       0		\\
    2,4056  &  2,3135  &  0,5562		\\
    1,4058  &  1,9500  &  0,6762		\\
    0,0511  &  -0,1980 &  0,1916		
	
	\end{pmatrix}  \nonumber
\end{equation}

f�r die vertikale Richtung $y$. (Die zweiten Spalten sind wieder in m,
die dritten in m$^2$.)

Der mit Matlab errechnete $\chi^2$-Fit ergibt

\begin{equation}
	\begin{pmatrix}
	a_{x,1}	\\
    a_{x,2}	\\
	a_{x,3}				
	\end{pmatrix}
	= 10^{-6} \cdot
	\begin{pmatrix}
	0,8417  \text{ m}^2 \\
	0,5456 \text{ m} \\
	0,9037			
	\end{pmatrix} \nonumber
\end{equation}

und

\begin{equation}
	\begin{pmatrix}
	a_{y,1}	\\
    a_{y,2}	\\
	a_{y,3}				
	\end{pmatrix}
	= 10^{-5} \cdot
	\begin{pmatrix}
	0,035  \text{ m}^2 \\
	-0,063 \text{ m} \\
	1,284			
	\end{pmatrix} \nonumber
\end{equation}

woraus (mit Fortpflanzung des $\chi^2$-Fit-Fehlers) folgt:

\begin{equation}
	\eps_{x,n} = (0,119 \pm 0,016)\text{ mm}\cdot\text{mrad} \nonumber
\end{equation}
und
\begin{equation}
	\eps_{y,n} = (0,35 \pm 0,02)\text{ mm}\cdot\text{mrad.} \nonumber
\end{equation}